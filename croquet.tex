\documentclass[a4paper]{article}

%% Language and font encodings
\usepackage[english]{babel}
\usepackage[utf8x]{inputenc}
\usepackage[T1]{fontenc}

%% Sets page size and margins
\usepackage[a4paper,top=3cm,bottom=2cm,left=3cm,right=3cm,marginparwidth=1.75cm]{geometry}

%% Useful packages
\usepackage{amsmath}
\usepackage{graphicx}
\usepackage[colorinlistoftodos]{todonotes}
\usepackage[colorlinks=true, allcolors=blue]{hyperref}
\usepackage{enumitem}
\setlist[enumerate,1]{start=0}

\title{Croquet}
\author{Ritzenthaler Tournament Rules}

\begin{document}
\maketitle
\begin{abstract}
This document offers clarification of commonly-misunderstood rules, not an explanation of the entire set of rules of croquet.
\end{abstract}

\begin{enumerate}
\item \textbf{If all players can agree on a set of rules prior to beginning the game - \textit{any} set of rules - then they may use those rules even if they violate these. If both teams cannot agree, the rules MUST fall back to these.}
\item Mallets are distributed in a first-come first-served order. Players must use the ball that matches the color of their mallet.
\item The exception to the above rule is in the championship match. In the championship match, those who came in first place in their previous matches must be allowed to choose mallets first. Then those who placed second will be allowed to choose mallets, and finally the third place runner-ups will be allowed to choose (if applicable). Between those who placed the same in their preliminary matches, mallets are chosen in a first-come, first-served order.
\item Typically, two or three preliminary matches are played depending on participation. To accommodate the possibility of two or three preliminaries, matches should continue until the last player is left standing, so that a strict hierarchy is formed.
\item A player may earn extra strokes on their turn by striking another player(s)'s ball(s) with their own ball - which awards two extra strokes - or by going through a hoop - which awards one extra stroke per hoop. However, a player cannot earn more than two extra strokes by striking another player's ball in a single stroke. This means that if a player makes a stroke and their ball hits two of their opponents' balls before coming to rest, they have two strokes remaining.
\item When a player's ball impacts another player's ball in the manner described above, they may either play the ball as it lies, or move the ball to anywhere within a mallet head's length from \textit{any} of the balls they impacted with that stroke.
\item The caveat to the above rule is that whenever a player's ball passes through a hoop, any extra strokes they may have accrued through striking players are stripped from them, and they cannot move their ball back to the ball of another player they impacted on their way to the hoop.
\item There is no limit to the number of additional strokes a player may earn by passing through hoops, but any additional strokes accrued in this manner are lost instantly when the player makes contact with a stake.
\item When a player hits a stake that does not cause them to become Dead or Poison, they gain one extra stroke.
\item If a player has impacted another player's ball and has already earned extra strokes for striking a player in the same stroke, then striking that same player's ball again will not award any extra strokes until passing through a hoop. So, for example, if player A hits player B's ball and does not manage to pass through the next hoop in their rotation within the two strokes provided to them, then proceeds to hit player B's ball again on their next turn, they do not gain any extra strokes. If, however, on player A's next turn they pass through a hoop and then use the extra stroke thereby accrued to strike player B's ball, they now have 2 strokes remaining in the turn.
\item If a player's ball passes through a hoop for any reason, on any player's turn, it counts as their having gone through the hoop - but they are not awarded an extra stroke unless it is their turn (and they are not Dead and did not already lose the game).
\item Once a player has made it around the entire play field and once he or she has struck the starting stake, he or she is now Dead. Dead players grant no bonus strokes when struck, and can gain no bonus strokes by any means - hitting other players or going through hoops - and may pass by any path to the stake at the opposite end of the field.
\item When a Dead player strikes the opposite stake, they are now Poison. Any player that is not Poison who's ball touches a Poison ball under any circumstances instantly loses the game. A Poison player that on their turn strikes another Poison player's ball eliminates that player. A Poison player instantly loses the game if their ball goes through a hoop at any point, for any reason, on any player's turn. Whenever a Poison player eliminates a player on their turn, they gain an additional stroke.
\item When a player hits a stake that makes them Dead or Poison, their turn is over, regardless of the number of strokes they have remaining. If the stake makes a player Dead or Poison, they must place their ball anywhere they choose that is exactly a full mallet handle's length away from the stake. Placing a ball in this way does NOT count as going through a hoop, even if the line between the ball and the stake happens to go through a hoop.
\item When a player strikes a stake that does NOT make them Dead or Poison, their turn is not over until they are out of strokes, and they must place the ball within one mallet HEAD length away from the stake before taking another stroke. Placing a ball in this way does NOT count as going through a hoop, even if the line between the ball and the stake happens to go through a hoop.
\end{enumerate}
\end{document}
