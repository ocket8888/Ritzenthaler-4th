\documentclass[a4paper]{article}

%% Language and font encodings
\usepackage[english]{babel}
\usepackage[utf8x]{inputenc}
\usepackage[T1]{fontenc}

%% Sets page size and margins
\usepackage[a4paper,top=3cm,bottom=2cm,left=3cm,right=3cm,marginparwidth=1.75cm]{geometry}

%% Useful packages
\usepackage{amsmath}
\usepackage{graphicx}
\usepackage[colorinlistoftodos]{todonotes}
\usepackage[colorlinks=true, allcolors=blue]{hyperref}
\usepackage{enumitem}
\setlist[enumerate,1]{start=0}

\title{Ladder Toss (Hillbilly Golf)}
\author{Ritzenthaler Tournament Rules}

\begin{document}
\maketitle
\begin{enumerate}
\item \textbf{If both teams can agree on a set of rules prior to beginning the game - \textit{any} set of rules - then they may use those rules even if they violate these. If both teams cannot agree, the rules MUST fall back to these.}
\item The team that tosses first is chosen by coin toss or a game of Rock-Paper-Scissors (best of ONE). The teammate that tosses first will be chosen by the team that is NOT tossing first. That is, if Team A wins the coin toss against Team B, then after choosing which teammate is on which side, Team B will decide which side tosses first.
\item Tosses take place in staggered order, first the first team tosses, then the other and so on until all ball pairs have been tossed.
\item After the first round of tossing, the team with the higher number of points tosses first. If there is a tie in points, the team that previously tossed first still tosses first.
\item When tossing, the player must not step beyond the front of the ladder struts on his or her side.
\item A ball pair can touch the ground without becoming "dead" in any sense, and a ball pair that bounces onto a ladder rung does score points or cancel points, as appropriate.
\item A ball pair that is wrapped around a rung scores points according to the point value of the rung (explained below), even if it uses one of the side posts for support. If two ball pairs of opposite teams are wrapped around the SAME rung, their points cancel. This is applicable even if the ball pair bounced off the ground (as per the previous rule) or if it dropped from a previous rung. Points are tallied after all ball pairs have come to rest with respect to the ladder at which they were tossed.
\item From top to bottom the points awarded for each ladder rung are as follows:
	\begin{itemize}
		\item 3
		\item 2
		\item 1
	\end{itemize}
\item A ball pair is considered "on" a rung if said rung prevents the balls from falling - even if they are actually wrapped around a side post, the rung that stops their descent is considered the rung on which they rest. If, somehow, a ball pair might be considered on more than one rung by this definition, then the rung with the higher point value is considered the rung on which it rests.
\item The game is over when exactly one team reaches exactly 21 points. If, after points are tallied after the end of a round of tossing, a team's score would exceed 21, it is reset to the score the team had at the beginning of the round. If two teams reach 21 points at the same time, both of their scores are reset to the scores they had at the beginning of the round, and play continues as normal.
\end{enumerate}

\end{document}
