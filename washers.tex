\documentclass[a4paper]{article}

%% Language and font encodings
\usepackage[english]{babel}
\usepackage[utf8x]{inputenc}
\usepackage[T1]{fontenc}

%% Sets page size and margins
\usepackage[a4paper,top=3cm,bottom=2cm,left=3cm,right=3cm,marginparwidth=1.75cm]{geometry}

%% Useful packages
\usepackage{amsmath}
\usepackage{graphicx}
\usepackage[colorinlistoftodos]{todonotes}
\usepackage[colorlinks=true, allcolors=blue]{hyperref}
\usepackage{enumitem}
\setlist[enumerate,1]{start=0}

\title{Washers}
\author{Ritzenthaler Tournament Rules}

\begin{document}
\maketitle
\begin{enumerate}
\item \textbf{If both teams can agree on a set of rules prior to beginning the game - \textit{any} set of rules - then they may use those rules even if they violate these. If both teams cannot agree, the rules MUST fall back to these.}
\item The team that tosses first is chosen by coin toss or a game of Rock-Paper-Scissors (best of ONE). The teammate that tosses first will be chosen by the team that is NOT tossing first. That is, if Team A wins the coin toss against Team B, then after choosing which teammate is on which side, Team B will decide which side tosses first.
\item When a team tosses, they toss all of their washers, one-by-one, before the other team tosses.
\item After the first round of tossing, the team that scored the most points in the previous rounds tosses first. If there was a tie in points, the team that previously tossed first still tosses first.
\item A washer scores points only for falling into a hole; a washer on the board is worth no points. Also, a washer which hits the ground is "dead" and incapable of scoring or canceling points. However, a washer which hits the ground can legally knock another washer off of the board.
\item Points are gained by tossing a washer such that it is not "dead" AND falls into one of the holes. From furthest away to closest, the point values of the holes are
\begin{itemize}
	\item 3
	\item 2
	\item 1
\end{itemize}
\item When a team lands a washer into a hole in which the other team has already landed a washer, the points they would have earned are instead subtracted from their opponents' score. For example, if Team A has landed a washer into the one-point hole, then Team B lands a washer into the SAME hole, then the new score is 0/0, not 1/1. At this point, another washer in the same hole scores points normally - that is, only a player who has a majority of washers in a hole may score points in that hole, and \textit{only} with the washer they have in excess over their opponent in the same hole.
\item Unlike Beanbag Toss and Ladder Toss, points are tallied as they are scored instead of at the end of the round. When points are scored that put a team's point total above 21, the earned points are subtracted from the point total rather than added. So if a team has 19 points and gets a washer in the three-point hole, their new point total is 16.
\item A team wins by both hitting exactly 21 points and getting all of their washers either in holes or on the board. If a team has exactly 21 points at the beginning of a round, a desirable, winning round would be to have all of the washers on the board but none in any holes. That is, it is not necessary to score on a winning round, but it \textit{is} required that there be no "dead" washers. (This is the most consistently-disregarded rule outside of the championship match)
\end{enumerate}
\end{document}
